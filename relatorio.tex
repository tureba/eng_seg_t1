\documentclass[12pt]{article}
\usepackage[T1]{fontenc}
\usepackage[utf8]{inputenc}
\usepackage[portuges]{babel}
\usepackage{hyperref}

\title{Relatório do Trabalho de\\Engenharia de Segurança\vspace{3cm}}
\author{
	\begin{tabular}{lcr}
		Arthur Nascimento & -- & nºUSP 5634455\\
		Gabriel Lima & -- & nºUSP 5744830\\
	\end{tabular}
}

\begin{document}

\maketitle
\pagebreak

\section{Algoritmo de Encriptação}

O algoritmo escolhido para este trabalho é similar à clássica cifra de César. Nesta cifra conhecida,
cada letra do alfabeto é substituída por uma que está a uma certa distância da original. No nosso algoritmo,
isso é feito para as consoantes e o deslocamento é a soma do primeiro, terceiro e quinto dígitos da chave.
As vogais são substituídas por um número (menos um) na ordem que aparecem na chave, ou seja, para uma chave
23154, a letra '\emph{a}' é substituída por 1, a letra '\emph{e}' é substituída por 2, '\emph{i}' por 0,
'\emph{o}' por 4 e '\emph{u}' por 3. Para as letras maiúsculas soma-se 5 ao número ao fazer a substituição.
Por este motivo, o nosso algoritmo não encripta corretamente números.

\section{Instruções}

Foram feitos dois programas que implementam o algoritmo. O programa que encripta a mensagem se chama \emph{crypt}
e o que decripta a mensagem se chama \emph{decrypt}. O código pode ser encontrado em
\url{http://github.com/tureba/eng\_seg\_t1}.

\subsection{Compilação}

A compilação dos programas, assim como da documentação, foi simplificada com o uso da ferramenta \emph{make},
como especificada no padrão {POSIX.1-2008} em \url{http://www.opengroup.org/onlinepubs/9699919799/}. O código
está no padrão mais recente da linguagem C, conhecido como ISO C99
(\url{http://www.open-std.org/JTC1/SC22/WG14/www/standards}), então exige um compilador suficientemente
atual. Como a maioria dos compiladores suportam este padrão, isso não restringe a adoção do software. Para
habilitar este suporte no GCC (\url{http://gcc.gnu.org/}), usa-se a opção \emph{-std=c99}.

Para se compilar os programas, digite:

\begin{verbatim}
	make progs
\end{verbatim}

Para compilar este relatório:

\begin{verbatim}
	make relatorio.pdf
\end{verbatim}

Para compilar tudo:

\begin{verbatim}
	make
\end{verbatim}

\subsection{Execução}

Seguindo a filosofia do UNIX\textregistered, os programas produzidos são filtros de dados e, portanto, lêem
da entrada padrão, fazem o seu processamento e então escrevem na saída padrão. Estes programas recebem exatamente
um argumento de linha de comando que é a chave de encriptação/decriptação.

Desta forma, para se encriptar um arquivo de texto \texttt{texto.txt} com uma chave hipotética 34521 executa-se:

\begin{verbatim}
	./crypt 34521 <texto.txt
\end{verbatim}

O texto encriptado será mostrado no terminal. Pode-se salvar o resultado em \texttt{resultado.txt}:

\begin{verbatim}
	./crypt 34521 <texto.txt >resultado.txt
\end{verbatim}

Para se testar a encriptação e a decriptação é possível canalizar os dados desta forma:

\begin{verbatim}
	./crypt 34521 <texto.txt | ./decrypt 34521
\end{verbatim}

O texto deve ser mostrado no terminal plenamente legível.


\section{Resultados com a Ferramenta de Criptoanálise}

Escolhemos uma ferramente de criptoanálise com interface web, encontrada no site da
Yellowpipe Internet Services (\url{http://www.yellowpipe.com/yis/tools/encrypter/}).
Esta ferramenta foi escolhida por ser de fácil utilização e entendimento, além de contar com
métodos de força bruta que eram suficientemente poderosos para decifrar o nosso algoritmo.

Utilizamos os seguintes mensagens originais:

\begin{verbatim}
	Engenharia de seguranca
	Arthur Nascimento
	Gabriel Luis
\end{verbatim}

Que, ao serem criptografados com a chave 14235, se tornaram, respectivamente:

\begin{verbatim}
	8ib3ic0m10 y3 n3b4m0ix
	5moc4m S0nx1h3io2
	L0wm13g Q41n
\end{verbatim}

A ferramenta web permite que escolha certos algoritmos de criptografia para tentar descriptografar
a mensagem. Após uma tentativa, uma série de possíveis mensagens é exibida. Por meio de inspeção visual
é possível dizer qual é a mensagem, se houve ou não algum sucesso, ou o quão próximo está disso. Para
nosso exemplo, utilizamos a opção Caesar Bruteforce. A saída gerada pela ferramenta foi, para cada um
dos exemplos, respectivamente:

Exemplo 1:

\begin{verbatim}
	ROT-1: 8jc3jd0n10 z3 o3c4n0jy0
	ROT-2: 8kd3ke0o10 a3 p3d4o0kz0
	ROT-3: 8le3lf0p10 b3 q3e4p0la0
	ROT-4: 8mf3mg0q10 c3 r3f4q0mb0
	ROT-5: 8ng3nh0r10 d3 s3g4r0nc0
	ROT-6: 8oh3oi0s10 e3 t3h4s0od0
	ROT-7: 8pi3pj0t10 f3 u3i4t0pe0
	ROT-8: 8qj3qk0u10 g3 v3j4u0qf0
	ROT-9: 8rk3rl0v10 h3 w3k4v0rg0
	ROT-10: 8sl3sm0w10 i3 x3l4w0sh0
	ROT-11: 8tm3tn0x10 j3 y3m4x0ti0
	ROT-12: 8un3uo0y10 k3 z3n4y0uj0
	ROT-13: 8vo3vp0z10 l3 a3o4z0vk0
	ROT-14: 8wp3wq0a10 m3 b3p4a0wl0
	ROT-15: 8xq3xr0b10 n3 c3q4b0xm0
	ROT-16: 8yr3ys0c10 o3 d3r4c0yn0
	ROT-17: 8zs3zt0d10 p3 e3s4d0zo0
	ROT-18: 8at3au0e10 q3 f3t4e0ap0
	ROT-19: 8bu3bv0f10 r3 g3u4f0bq0
	ROT-20: 8cv3cw0g10 s3 h3v4g0cr0
	ROT-21: 8dw3dx0h10 t3 i3w4h0ds0
	ROT-22: 8ex3ey0i10 u3 j3x4i0et0
	ROT-23: 8fy3fz0j10 v3 k3y4j0fu0
	ROT-24: 8gz3ga0k10 w3 l3z4k0gv0
	ROT-25: 8ha3hb0l10 x3 m3a4l0hw0
\end{verbatim}

Exemplo 2:

\begin{verbatim}
	ROT-1: 5npd4n T0oy1i3jp2
	ROT-2: 5oqe4o U0pz1j3kq2
	ROT-3: 5prf4p V0qa1k3lr2
	ROT-4: 5qsg4q W0rb1l3ms2
	ROT-5: 5rth4r X0sc1m3nt2
	ROT-6: 5sui4s Y0td1n3ou2
	ROT-7: 5tvj4t Z0ue1o3pv2
	ROT-8: 5uwk4u A0vf1p3qw2
	ROT-9: 5vxl4v B0wg1q3rx2
	ROT-10: 5wym4w C0xh1r3sy2
	ROT-11: 5xzn4x D0yi1s3tz2
	ROT-12: 5yao4y E0zj1t3ua2
	ROT-13: 5zbp4z F0ak1u3vb2
	ROT-14: 5acq4a G0bl1v3wc2
	ROT-15: 5bdr4b H0cm1w3xd2
	ROT-16: 5ces4c I0dn1x3ye2
	ROT-17: 5dft4d J0eo1y3zf2
	ROT-18: 5egu4e K0fp1z3ag2
	ROT-19: 5fhv4f L0gq1a3bh2
	ROT-20: 5giw4g M0hr1b3ci2
	ROT-21: 5hjx4h N0is1c3dj2
	ROT-22: 5iky4i O0jt1d3ek2
	ROT-23: 5jlz4j P0ku1e3fl2
	ROT-24: 5kma4k Q0lv1f3gm2
	ROT-25: 5lnb4l R0mw1g3hn2
\end{verbatim}

Exemplo 3:

\begin{verbatim}
	ROT-1: M0xn13h R41o
	ROT-2: N0yo13i S41p
	ROT-3: O0zp13j T41q
	ROT-4: P0aq13k U41r
	ROT-5: Q0br13l V41s
	ROT-6: R0cs13m W41t
	ROT-7: S0dt13n X41u
	ROT-8: T0eu13o Y41v
	ROT-9: U0fv13p Z41w
	ROT-10: V0gw13q A41x
	ROT-11: W0hx13r B41y
	ROT-12: X0iy13s C41z
	ROT-13: Y0jz13t D41a
	ROT-14: Z0ka13u E41b
	ROT-15: A0lb13v F41c
	ROT-16: B0mc13w G41d
	ROT-17: C0nd13x H41e
	ROT-18: D0oe13y I41f
	ROT-19: E0pf13z J41g
	ROT-20: F0qg13a K41h
	ROT-21: G0rh13b L41i
	ROT-22: H0si13c M41j
	ROT-23: I0tj13d N41k
	ROT-24: J0uk13e O41l
	ROT-25: K0vl13f P41m
\end{verbatim}

Note que, para todos os exemplos, a ferramenta chega muito próxima do resultado verdadeiro através do
algoritmo ROT5, faltando apenas substituir os números pelas vogais correspondentes e poucas consoantes.
Portanto, apenas a força bruta, neste caso, seria o suficiente para quebrar a criptografia, bastaria uma
pequena inspeção para saber qual é a verdadeira mensagem.

\end{document}
